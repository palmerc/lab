\documentclass[12pt,twoside,letterpaper]{report}
\usepackage{graphicx}
\setlength\topmargin{0in}
\setlength\headheight{0in}
\setlength\headsep{0in}
\setlength\textheight{9.0in}
\setlength\textwidth{6.5in}
\setlength\oddsidemargin{0in}
\setlength\evensidemargin{0in}
\setlength{\parskip}{12pt}
\pagestyle{plain}
\raggedright
\title{mTrader Server Protocol}
\author{Cameron Lowell Palmer}
\date{January 2010}
\begin{document}
\maketitle

\section*{Protocol Description}
The mTrader protocol is a TCP ASCII-encoded line-oriented communication protocol. Messages are transmitted without compression or encryption.

The server at wireless.theonlinetrader.com can be connected to on port 7780.

End-of-line (EOL) is defined as carriage-return "\textbackslash r" line-feed "\textbackslash n" or 0x0d 0x0a in hexidecimal.

Each line of a request or response consists of a verb and the associated data separated by colon with the exception of the HTTP OK header.

Requests begin with an action verb and any other necessary verbs to complete the request, each on a  separate line. A full request must be completed in ? seconds or the request will timeout. A trailing blank line indicates the end of the request.

Responses are separated by two blank lines. 

\section*{Issues With mTrader Server}
\begin{itemize}
\item IP header checksums from server are always zero. This is technical violation of the IP protocol.
\item A user session can be hijacked without a password. (eg. Authorization: username)
\item User data is plain text including login credentials.
\item Update messages occur even when not being monitored, and even when messages are empty.
\item Data is unencrypted, and uncompressed.
\end{itemize}

\chapter*{Basic Authentication and Session Handling}
\section*{Header}
A header always follows a request. In either streaming or non-streaming mode you will receive:
\begin{verbatim}
HTTP/1.1 200 OK
Server: MMS
\end{verbatim}

If the connection is not in streaming mode the response will contain a content length message indicating the number of bytes in the response that will follow the header.
\begin{verbatim}
HTTP/1.1 200 OK
Server: MMS
Content-length: xxxx
\end{verbatim}

\section*{Login Request}
Typical format of a login request:
\begin{verbatim}
Action: Login
Authorization: username/password
Platform: ...
Client: ...
VerType: x.x
ConnType: {Socket,Legacy}
Streaming: {0,1}
QFields: ...
\end{verbatim}

Options definitions:
$$
\begin{array}{lll}
\texttt{Action} & \texttt{Login action} & \texttt{Required} \\
\texttt{Authorization} & \texttt{Username and Password seperated by "/"} & \texttt{Required} \\
\texttt{Platform} & \texttt{Name of the platform} & \texttt{optional} \\
\texttt{Client} & \texttt{Name of the client} & \texttt{optional} \\
\texttt{VerType} & \texttt{Version numer of client} & \texttt{optional} \\
\texttt{ConnType} & \texttt{Can be equal to "Socket", or "Legacy"} & \texttt{optional} \\
\texttt{Streaming} & \texttt{Can be equal to "0" (false) or "1" (true)} & \texttt{defaults to "0"} \\
\texttt{QFields} & \texttt{See section on QFields} & \texttt{defaults to empty}
\end{array}
$$

\section*{Login Response}
Successful login response:
\begin{verbatim}
HTTP/1.1 200 OK
Server: MMS
Content-Length: xxxx

Request: login/OK
Version: x.xx.xx
DLURL: 
ServerIP: xxx.xxx.xxx.xxx
User: username
Securities: ...
Exchanges: ...
NewsFeeds: ...
\end{verbatim}

Failed login responses:
\begin{itemize}
\item Request: login/failed.UsrPwd
\item Request: login/failed.DeniedAccess
\end{itemize}

\section*{Kickout}
When a user is kicked off because they logged in from another terminal while streaming is enabled, they will receive:

\begin{verbatim}
HTTP/1.1 200 OK
Server: MMS

Request: q
Kickout: 1
\end{verbatim}

Then their connection is closed.

\section*{Logout}
The J2ME mTrader client does not issue a logout command to the server on logout.

An example of a valid Logout command is as follows:
\begin{verbatim}
Action: Logout

HTTP/1.1 200 OK
Server: MMS
Content-Length: 4
\end{verbatim}
Followed by the server closing the connection.

\section*{Streaming}
To enable streaming, the client version must be greater than or equal to 2.2 and streaming set to be not equal to zero. Streaming can only be enabled during the initial login, and streaming mode can only be changed by disconnecting and reconnecting.

For example:
\begin{verbatim}
VerType: 2.2
Streaming: 1
\end{verbatim}

A QFields request must be provided during login or you will receive only an empty response whenever a value in you stocks list is updated. Or maybe this is the keep-alive set to every 15 seconds...

To change the list of streamed QFields after starting:
\begin{verbatim}
Action: q
Authorization: username
QFields: ...
\end{verbatim}

The streamed updates will not begin with an HTTP header. Only new actions will receive a header preamble.

\chapter*{Handling Different Request Types}
\section*{Actions Descriptions}
Valid Actions are:
\begin{itemize}
\item LOGIN Login Request
\item Q Quote Request
\item PORTREQ Portfolio Request
\item ORDERSREQ Orders Request
\item ADDSEC Add Security
\item REMSEC Remove Security
\item TRADE Trade Request
\item TRADECHANGE Trade Change Request
\item TRADECANCEL Trade Cancel Request
\item STATDATA Security Info Request
\item LOGOUT Logout Request
\item CHART Chart Request
\item HISTTRADES Trades History Request
\item NEWSLIST News List Request
\item NEWSLISTFEEDS News List Feeds Request
\item NEWSBODY News Body Request
\item POSITIONINFO Position Info Request
\item ERROR Error Message
\end{itemize}

\section*{QFields Descriptions}
QFields are specified using a semicolon seperated list of values:
\begin{itemize}
\item l - last trade
\item cp - percent change
\item b - bid price
\item a - ask price
\item av - ask volume
\item bv - bid volume
\item c - change
\item h - high
\item lo - low
\item o - open
\item v - volume
\end{itemize}

An example QField:
\begin{verbatim}
QFields: l;cp;c
\end{verbatim}

In streaming mode, an empty or unspecified QFields will result in updates without information.

\section*{The quote response}
Following a successful quote action you will receive:

\begin{enumerate}
\item HTTP Header if not streaming
\item Request: q
\item Quotes: Exchange/Ticker;QFields
\item Two blank lines
\end{enumerate}

The quote line is comprised of the exchage number and ticker symbol and the data requested via the QField. The data will be in the order specified in the QField separated by semicolons.

\section*{Add a security}
Both success and failure begin with the HTTP response header.

Successfully adding a security results in the response: 
\begin{verbatim}
Request: addSec/OK
SecInfo: ???
\end{verbatim}

Possible failed to add repsonses:
\begin{itemize}
\item Request: addSec/failed.AlreadyExists
\item Request: addSec/failed.NoSuchSec
\end{itemize}

\section*{Remove a security}
Both success and failure begin with the HTTP response header.

Success results in:
\begin{verbatim}
Request: remSec/OK
\end{verbatim}

Possible failed to remove repsonses:
\begin{itemize}
\item Request: remSec/failed.CouldNotDelete
\end{itemize}

\section*{News Requests}
\begin{verbatim}
Action: NewsListFeeds
Authorization: username
NewsFeeds: Oslo News [OBI]
Days: 30
MaxCount: 30
\end{verbatim}

\section*{Stock Information Request}
\begin{verbatim}
Action: StatData
Authorization: username
SecOid: 18177/TEL
Language: EN
\end{verbatim}


\end{document}

\documentclass[12pt,twoside,letterpaper]{article}
\setlength\topmargin{0in}
\setlength\headheight{0in}
\setlength\headsep{0in}
\setlength\textheight{9.0in}
\setlength\textwidth{6.5in}
\setlength\oddsidemargin{0in}
\setlength\evensidemargin{0in}
\setlength{\parskip}{12pt}
\pagestyle{empty}
\raggedright
\begin{document}
\section*{Maximum Independent Set in a Tree}
For any vertex $x \in T$. Let $T_x$ denote the subtree rooted at $x$. Let $w(x)$ be the weight of $x$. Let $W_x$ be the value of an optimal solution if we force $x$ into the solution. Let $W_x^\prime$ be the value for $T_x$ if we force $x$ out of the solution. If $O_x$ is the optimal value for $T_x$ then:

If $x$ is not a leaf,
\begin{equation}
O_x = \max(W_x,W_x^\prime)
\end{equation}
Let $C(x)$ be the children of $x$,
\begin{equation}
W_x = w(x) + \sum_{y \in C(x)} W_y^\prime
\end{equation}
\begin{equation}
W_x^\prime = \sum_{y \in C(x)} O_y
\end{equation}

If $x$ is a leaf,
$$W_x= W(x)$$
$$W_x^\prime = O$$

The algorithm is post order in which recurrence relations (1), (2), (3) are used to compute $O_x, W_x, W_x^\prime$ for all $x \in V$.

Time complexity is $\textrm{O}(n)$ where $n = |v|$.
\end{document}
\documentclass{article}
\usepackage{amsmath, amsthm, amsfonts}

\begin{document}
\section{Linear Dependence}

Column's of B are linearly dependent $b_1^{\rightharpoonup},
b_2^{\rightharpoonup}, ..., b_p^{\rightharpoonup}$.

There are scalars $c_1, ..., c_p$ not all zero, such that
$c_1b_1^{\rightharpoonup}+c_2b_2^{\rightharpoonup}+...+c_pb_p^{\rightharpoonup}
=0^{\rightharpoonup}$

\begin{proof}
$A(BC)=(AB)C$

\[ C=[c_1^{\rightharpoonup} ... c_p^{\rightharpoonup}] \]\\
Then,\\
\[ BC=[Bc_1 ... Bc_p] \]\\
so,
\begin{align*}
A(BC) & =[ABc_1 ... ABc_p]\\
& =[(AB)c_1^{\rightharpoonup} ... (AB)c_p^{\rightharpoonup}]\\
& =ABC
\end{align*}
\end{proof}

\begin{quote}
In general, $AB\not=BA$.\\
If $AB=AC$ it does generally not follow that $B=C$.\\
If $AB=BA$ we say A and B commute. In the product $AB$, we say A is
right-multiplied by B, and B is left-multiplied by A.
\end{quote}


\section{Powers of a Matrix if it is by Square or $m{\times}m$}
By convention $A^0=I$

\[ A^1=\left[
       \begin{array}{cc}
         1 & 1 \\
         2 & 0 \\
       \end{array}
     \right]
\]
\[
   A^2=\left[
       \begin{array}{cc}
         1 & 1 \\
         2 & 0 \\
       \end{array}
     \right]
     \cdot
     \left[
       \begin{array}{cc}
         1 & 1 \\
         2 & 0 \\
       \end{array}
     \right]
     =
     \left[
       \begin{array}{cc}
         (1+2) & 1 \\
         (2+0) & 2 \\
       \end{array}
     \right]
     \]
     \[
    A^3=\left[
       \begin{array}{cc}
         3 & 1 \\
         2 & 2 \\
       \end{array}
     \right]
     \cdot
     \left[
       \begin{array}{cc}
         1 & 1 \\
         2 & 0 \\
       \end{array}
     \right]
     =
     \left[
       \begin{array}{cc}
         5 & 3 \\
         6 & 2 \\
       \end{array}
     \right]
\]

\section{The Transpose of a Matrix}
If A is a $m{\times}n$ matrix, the transpose of A, denoted $A^T$ is
the $n{\times}m$ matrix whose columns are the rows of A, in the same
order.

\emph{example} -
\begin{enumerate}
\item\[ A = \left[
       \begin{array}{ccc}
         1 & -1 & 5 \\
         2 & 0 & 6 \\
       \end{array}
     \right]
     \Rightarrow
     \left[
       \begin{array}{cc}
         1 & 2 \\
         -1 & 0 \\
         5 & 6 \\
       \end{array}
     \right]
\]

\item\[ A=\left[
          \begin{array}{cc}
             a & b \\
             c & d \\
          \end{array}
          \right]
          \Rightarrow
          A^T=
          \left[
          \begin{array}{cc}
             a & c \\
             b & d \\
          \end{array}
          \right]
          \]
\item\[ A=I_{12} \Rightarrow A^T = A \]

\item\[ A=\left[
          \begin{array}{ccc}
             1 & 3 & 2 \\
             3 & 4 & 3 \\
             2 & 3 & 2 \\
          \end{array}
          \right]
          {\Rightarrow}A^T=A \]
          \emph{note} - Symmetric about the diagonal
\end{enumerate}

\emph{Properties}
\begin{enumerate}
  \item $(A^T)^T=A$
  \item $(A+B)^T=A^T+B^T$
  \item $(rA)^T=r(A^T)$
  \item $(AB)^T=B^TA^T$
\end{enumerate}

\emph{Example} -
\[ A=\left[ \begin{array}{cc}
               1 & 3 \\
               0 & 1 \\
            \end{array}\right]
   B=\left[ \begin{array}{cc}
               -1 & 2 \\
               1 & -3 \\
            \end{array}\right]
\]
\[ AB=\left[ \begin{array}{cc}
                2 & -7 \\
                1 & -3 \\
             \end{array} \right]
\]
\[ (AB)^T=\left[ \begin{array}{cc}
                2 & 1 \\
                -7 & -3 \\
             \end{array} \right]
\]
\[ A^T=\left[ \begin{array}{cc}
                1 & 0 \\
                3 & 1 \\
             \end{array} \right],
   B^T=\left[ \begin{array}{cc}
                -1 & 1 \\
                2 & -3 \\
             \end{array} \right]
\]
\[ A^TB^T=\left[ \begin{array}{cc}
                -1 & 1 \\
                -1 & 0 \\
             \end{array} \right]
             \not=(AB)^T
\]
\[ B^TA^T=\left[ \begin{array}{cc}
                2 & 1 \\
                -7 & 3 \\
             \end{array} \right]
             =(AB)^T
\]

\section{The Inverse of a Matrix}
A $n{\times}n$ matrix A is invertible if there exists another
$n{\times}n$ matrix C such that $AC=CA=I$.

In this case, we call C the inverse of A, denoted $A^{-1}$.

The inverse is unique:\\
If B is an inverse of A, then $B=BI=B(AC)=(BA)C=IC=C$. So,
$AA^{-1}=A^{-1}A=I$. A matrix that is not invertible is called
singular; an invertible matrix is called non-singular.

\emph{Example}
\[ A=\left[ \begin{array}{cc}
                -1 & 1 \\
                5 & -4 \\
             \end{array} \right],
   C=\left[ \begin{array}{cc}
                4 & 1 \\
                5 & 1 \\
             \end{array} \right]
\]
Show that $C=A^{-1}$ We must show that $AC=I$ and $CA=I$ A must be
square to have an inverse.
\[ A=\left[ \begin{array}{ccc}
                1 & 0 & 2 \\
                1 & 1 & 1 \\
             \end{array} \right],
   B^T=\left[ \begin{array}{cc}
                3 & 2 \\
                -2 & 0 \\
                -1 & -1 \\
             \end{array} \right]
\]
Then,
\[ AC=\left[ \begin{array}{cc}
                1 & 0 \\
                1 & 1 \\
             \end{array} \right], CA \not= I_3
\]

Inverse of a $2{\times}2$ matrix.
Let \[ A=\left[ \begin{array}{cc}
                a & b \\
                c & d \\
             \end{array} \right]
\]
\begin{itemize}
\item If $ad-bc\not=0$ then A is invertible, and
\[ A^{-1}=\frac{1}{ad-bc}\cdot
\left[ \begin{array}{cc}
                d & -b \\
                -c & a \\
       \end{array} \right]
\]
\item If $ad-bc=0$, then A is not invertible
\[ A^{-1}=\frac{1}{ad-bc}\cdot
\left[ \begin{array}{cc}
                d & -b \\
                -c & a \\
       \end{array} \right]
       =I
\]
Thus a $2{\times}2$ matrix A is invertible iff $detA\not=0$.
\end{itemize}

Let A be an invertible $n{\times}n$ matrix. For every $b^{\rightharpoonup}\epsilon\mathbb{R}^n$, the equation $Ax^{\rightharpoonup}=b^{\rightharpoonup}$ has a unique solution, $x^{\rightharpoonup}=A^{-1}b^{\rightharpoonup}$.

\begin{proof}
$x^{\rightharpoonup}=A^{-1}b^{\rightharpoonup}$ is a solution since $Ax^{\rightharpoonup}=A(A^{-1}b^{\rightharpoonup}=(AA^{-1})b^{\rightharpoonup}=Ib^{\rightharpoonup}=b^{\rightharpoonup}$.

If $u^{\rightharpoonup}$ is a solution, then
\begin{align*}
Au^{\rightharpoonup} & =b^{\rightharpoonup}
Then A^{-1}(Au^{\rightharpoonup}) & =A^{-1}b^{\rightharpoonup}
(A^{-1}A)u^{\rightharpoonup} & =A^{-1}b^{\rightharpoonup}
Iu^{\rightharpoonup} & = x^{\rightharpoonup}
u^{\rightharpoonup} & =x^{\rightharpoonup}
\end{align*}
Thus the solution is unique
\end{proof}

\emph{example} -
\[ A=\left[ \begin{array}{cc}
                5 & 3 \\
                2 & 1 \\
       \end{array} \right]
\]

Use the inverse of A to solve the system
\begin{align*}
5x_1+3x_2 & =2\\
2x_1+x_2 & =-6
\end{align*}


\[ A^{-1}=\frac{1}{5{\cdot}1-3{\cdot}2}\cdot\left[ \begin{array}{cc}
                                                   1 & -3 \\
                                                   -2 & 5 \\
                                                   \end{array} \right]
=-1\left[ \begin{array}{cc}
          1 & -3 \\
          -2 & 5 \\
          \end{array} \right]
=\left[ \begin{array}{cc}
          -1 & 3 \\
          2 & -5 \\
          \end{array} \right] \]
\[
Then,
x^{\rightharpoonup}=A^{-1}\cdot\left[ \begin{array}{c}
                                     2 \\
                                     -6 \\
                                     \end{array}\right]
=\left[ \begin{array}{cc}
          -1 & 3 \\
          2 & -5 \\
          \end{array} \right]\cdot
\left[ \begin{array}{c}
       2 \\
       -6 \\
       \end{array}\right]
=\left[ \begin{array}{c}
       -20 \\
       34 \\
       \end{array}\right] \]
\[ x_1=-20, x_2=34 \]
\textbf{HOMEWORK}\\
$\S2.1: 27,28,33\\
\S2.2: 1,3,4,5,6,7,8,13$
\end{document}
